\documentclass[12pt]{article}
\setlength{\oddsidemargin}{0in}
\setlength{\evensidemargin}{0in}
\setlength{\textwidth}{6.5in}
\setlength{\parindent}{0in}
\setlength{\parskip}{\baselineskip}

\usepackage{amsmath,amsfonts,amssymb}
\usepackage{graphicx}  
\usepackage{pythonhighlight}

\title{CS206p Assignment 2}

\begin{document}

CS206p Assignment 2\hfill  \\
Jingcheng Lu (42326170)

\hrulefill

\begin{enumerate}
\item Question 1
  \begin{enumerate}
  \item 1.A
  \begin{equation}
  \begin{split}
  \label{eq1.1}
{\varepsilon _r}(f({x_0})) &= \frac{{\left| {f({x_0}) - f(t)} \right|}}{{f(t)}} \\
&= \frac{{\left| {f(t \times (1 + {\delta _0})) - {t^2}} \right|}}{{{t^2}}} \\
&= \frac{{\left| {{t^2} \cdot (1 + 2 \cdot {\delta _0} + {\delta _0}^2) - {t^2}} \right|}}{{{t^2}}} \\
&= 2 \cdot {\delta _0} + {\delta _0}^2 \\
&\approx 2 \cdot {\delta _0}
  \end{split}
  \end{equation}
  
  \item 1.B
  \begin{equation}
  \begin{split}
  \label{eq1.2}
{x_k} &= {t^{{2^k}}} \cdot (1 + {\delta _k})\\
 &= (1 + {\delta _0}) \cdot f({x_{k - 1}})\\
 &= (1 + {\delta _0}) \cdot {x_{k - 1}}^2\\
 &= (1 + {\delta _0}) \cdot {t^{{2^k}}} \cdot {(1 + {\delta _{k-1}})^2}
  \end{split}
  \end{equation}  
  Because of \eqref{eq1.2}, we can get:
  \begin{equation}
  \begin{split}
  \label{eq1.3} 
 &(1 + {\delta _k}) = (1 + {\delta _0}) \cdot {(1 + {\delta _{k - 1}})^2}\\
 &\to 2 \cdot {\delta _{k - 1}} + {\delta _0} + \varepsilon  = {\delta _k}\\
 &\to 2 \cdot {\delta _{k - 1}} + E = {\delta _k} + E - {\delta _0} - \varepsilon \\
 &\to \left| {{\delta _k}} \right| \le \left| {2 \cdot {\delta _{k - 1}}} \right| + E
  \end{split}
  \end{equation} 
  Based on \eqref{eq1.3}, for n=1, \[\left| {{\delta _1}} \right| \le 2 \cdot \left| {{\delta _0}} \right| + E\].
  If for n=k-1: \[\left| {{\delta _{k - 1}}} \right| \le {2^{k - 1}} \cdot \left| {{\delta _0}} \right| + ({2^{k - 1}} - 1) \cdot E\]
  So for n=k, we have:
  \begin{equation}
  \begin{split}
  & \left| {{\delta _k}} \right| \le 2 \cdot \left| {{\delta _{k - 1}}} \right| + E\\
 &= 2 \cdot ({2^{k - 1}} \cdot \left| {{\delta _0}} \right| + ({2^{k - 1}} - 1) \cdot E) + E\\
 &= {2^k} \cdot \left| {{\delta _0}} \right| + ({2^k} - 1) \cdot E
  \end{split}
  \end{equation} 
  \end{enumerate}
 \newpage

\item Question 2
  \begin{enumerate}
  \item 2.A
  \begin{equation}
\begin{split}
  \label{eq2.1} 
{\varepsilon _r}(f({x_0})) &= \frac{{\left| {f({x_0}) - f(t)} \right|}}{{f(t)}}\\ 
&\approx \frac{{\left| {f'(t)} \right|}}{{f(t)}} \cdot ({x_0} - t) \\
&= \frac{{\frac{1}{2} \cdot {t^{\frac{1}{2} - 1}} \cdot t \cdot {\delta _0}}}{{{t^{\frac{1}{2}}}}} \\
&= \frac{1}{2} \cdot {\delta _0}
\end{split}
\end{equation}

  \item 2.B \\
  Because of:
  \begin{equation}
  \begin{split}
  \label{eq2.2} 
{X_k} &= \sqrt {{t^{{2^{1 - k}}}} \cdot (1 + {\delta _{k - 1}})}  \\
  &= {t^{{2^{ - k}}}} \cdot \sqrt {(1 + {\delta _{k - 1}})}  \\
  &\approx {t^{{2^{ - k}}}} \cdot \sqrt {(1 + {\delta _{k - 1}} + \frac{1}{4}{\delta _{k - 1}}^2)} \\
 &= {t^{{2^{ - k}}}} \cdot (1 + \frac{1}{2}{\delta _{k - 1}})\\
 &= {t^{{2^{ - k}}}} \cdot (1 + {\delta _k})
\end{split}
  \end{equation}
  We can get:
    \begin{equation}
  \begin{split}
  \label{eq2.3} 
  \left| {{\delta _k}} \right| \le &\left| {\frac{1}{2}{\delta _{k - 1}}} \right| + E
\end{split}
  \end{equation}
  Based on \eqref{eq2.3}, for n=1, \[\left| {{\delta _1}} \right| \le \frac{1}{{{2^1}}} \cdot \left| {{\delta _0}} \right| + E\]
  If for n=k-1: 
  \[\left| {{\delta _{k - 1}}} \right| \le \frac{1}{{{2^{k - 1}}}} \cdot \left| {{\delta _0}} \right| + (2 - \frac{1}{{{2^{k - 2}}}}) \cdot E\]
  So for n=k, we have:
      \begin{equation}
  \begin{split}
  \label{eq2.3} 
&\left| {{\delta _k}} \right| \le \frac{1}{2} \cdot \left| {{\delta _{k - 1}}} \right| + E\\
 &\le \frac{1}{2} \cdot (\frac{1}{{{2^{k - 1}}}} \cdot \left| {{\delta _0}} \right| + (2 - \frac{1}{{{2^{k - 2}}}}) \cdot E) + E\\
 &= \frac{1}{{{2^k}}} \cdot \left| {{\delta _0}} \right| + (1 - \frac{1}{{{2^{k - 1}}}}) \cdot E + E
  \end{split}
  \end{equation}
    \end{enumerate}
\newpage

\item Question 3
  \begin{enumerate}
  \item 3.A \\
The function exp1 is for the experiment 1, the other exp2 is for the experiment 2.
  \begin{python}
import numpy as np
import math

alpha = np.float64(2.37e-7);

def computeX(x0,n):
    xn = x0;
    for n in range(0, n):
        xn = math.pow(xn,2);
    return xn;

def computeY(y0,n):
    yn = y0;
    for n in range(0, n):
        yn = math.sqrt(yn);
    return yn;

def exp1():
    x0 = 1 + alpha;
    t = 1 + alpha;
    for n in range(1, 31):
        xn = computeX(x0,n);
        errx = xn - t;
        y0 = xn;
        yn = computeY(y0,n);
        erry = yn - t;
        print("n: ",n,"; errx: ",errx,"; erry: ",erry);
    return 0;

def exp2():
    y0 = 1 + alpha;
    t = 1 + alpha;
    for n in range(1, 31):
        yn = computeY(y0, n);
        erry = yn - t;
        x0 = yn;
        xn = computeX(x0,n);
        errx = xn - t;
        print("n: ",n,"; errx: ",errx,"; erry: ",erry);
    return 0;
  \end{python}
The result in experiment1 is\\
('n: ', 1, '; errx: ', 2.3700005624682774e-07, '; erry: ', 0.0)\\
('n: ', 2, '; errx: ', 7.1100033727233836e-07, '; erry: ', 0.0)\\
('n: ', 3, '; errx: ', 1.6590015732287355e-06, '; erry: ', 0.0)\\
('n: ', 4, '; errx: ', 3.5550067414291675e-06, '; erry: ', 0.0)\\
('n: ', 5, '; errx: ', 7.347027862314448e-06, '; erry: ', 0.0)\\
('n: ', 6, '; errx: ', 1.493111324224472e-05, '; erry: ', 0.0)\\
('n: ', 7, '; errx: ', 3.0099456556298421e-05, '; erry: ', 0.0)\\
('n: ', 8, '; errx: ', 6.0436833413168856e-05, '; erry: ', 0.0)\\
('n: ', 9, '; errx: ', 0.00012111434814054967, '; erry: ', 0.0)\\
('n: ', 10, '; errx: ', 0.00024248042243080192, '; erry: ', 0.0)\\
('n: ', 11, '; errx: ', 0.00048525675660893164, '; erry: ', 0.0)\\
('n: ', 12, '; errx: ', 0.00097098621740565605, '; erry: ', 0.0)\\
('n: ', 13, '; errx: ', 0.0019431527093494161, '; erry: ', 0.0)\\
('n: ', 14, '; errx: ', 0.0038903191822612371, '; erry: ', 0.0)\\
('n: ', 15, '; errx: ', 0.0077960117919297911, '; erry: ', 0.0)\\
('n: ', 16, '; errx: ', 0.015653042079085244, '; erry: ', 0.0)\\
('n: ', 17, '; errx: ', 0.031551346304098393, '; erry: ', 0.0)\\
('n: ', 18, '; errx: ', 0.064098432017192231, '; erry: ', 0.0)\\
('n: ', 19, '; errx: ', 0.1323057404041601, '; erry: ', 0.0)\\
('n: ', 20, '; errx: ', 0.28211658946519047, '; erry: ', 0.0)\\
('n: ', 21, '; errx: ', 0.64382331970517148, '; erry: ', 0.0)\\
('n: ', 22, '; errx: ', 1.7021556485788403, '; erry: ', 0.0)\\
('n: ', 23, '; errx: ', 6.3016461929683665, '; erry: ', 0.0)\\
('n: ', 24, '; errx: ', 52.314040351269789, '; erry: ', 0.0)\\
('n: ', 25, '; errx: ', 2841.3869236106784, '; erry: ', 0.0)\\
('n: ', 26, '; errx: ', 8079162.4248600332, '; erry: ', 0.0)\\
('n: ', 27, '; errx: ', 65272881645598.922, '; erry: ', 0.0)\\
('n: ', 28, '; errx: ', 4.2605490783204951e+27, '; erry: ', 0.0)\\
('n: ', 29, '; errx: ', 1.815227844877762e+55, '; erry: ', 0.0)\\
('n: ', 30, '; errx: ', 3.2950521288195639e+110, '; erry: ', 0.0)\\
The result in experiment2 is\\
('n: ', 1, '; errx: ', -2.2204460492503131e-16, '; erry: ', -1.185000071401987e-07)\\
('n: ', 2, '; errx: ', -2.2204460492503131e-16, '; erry: ', -1.7775000538122754e-07)\\
('n: ', 3, '; errx: ', -2.2204460492503131e-16, '; erry: ', -2.0737500316947433e-07)\\
('n: ', 4, '; errx: ', -2.2204460492503131e-16, '; erry: ', -2.2218750173053081e-07)\\
('n: ', 5, '; errx: ', -2.2204460492503131e-16, '; erry: ', -2.2959375090003675e-07)\\
('n: ', 6, '; errx: ', -7.3274719625260332e-15, '; erry: ', -2.3329687559581203e-07)\\
('n: ', 7, '; errx: ', -7.3274719625260332e-15, '; erry: ', -2.3514843783267736e-07)\\
('n: ', 8, '; errx: ', -3.5749181392930041e-14, '; erry: ', -2.3607421906213233e-07)\\
('n: ', 9, '; errx: ', -9.2592600253738055e-14, '; erry: ', -2.3653710967685981e-07)\\
('n: ', 10, '; errx: ', -2.0627943797535409e-13, '; erry: ', -2.3676855498422356e-07)\\
('n: ', 11, '; errx: ', -2.0627943797535409e-13, '; erry: ', -2.3688427752688312e-07)\\
('n: ', 12, '; errx: ', -2.0627943797535409e-13, '; erry: ', -2.3694213879821291e-07)\\
('n: ', 13, '; errx: ', -2.0627943797535409e-13, '; erry: ', -2.369710694338778e-07)\\
('n: ', 14, '; errx: ', -2.0627943797535409e-13, '; erry: ', -2.3698553475171025e-07)\\
('n: ', 15, '; errx: ', -2.0627943797535409e-13, '; erry: ', -2.3699276741062647e-07)\\
('n: ', 16, '; errx: ', -7.48223705215878e-12, '; erry: ', -2.3699638385110688e-07)\\
('n: ', 17, '; errx: ', -7.48223705215878e-12, '; erry: ', -2.3699819196032479e-07)\\
('n: ', 18, '; errx: ', -3.6586067508892484e-11, '; erry: ', -2.3699909612595604e-07)\\
('n: ', 19, '; errx: ', -9.4793728422359891e-11, '; erry: ', -2.3699954820877167e-07)\\
('n: ', 20, '; errx: ', -2.1120905024929471e-10, '; erry: ', -2.3699977425017948e-07)\\
('n: ', 21, '; errx: ', -4.4403969390316433e-10, '; erry: ', -2.3699988727088339e-07)\\
('n: ', 22, '; errx: ', -4.4403969390316433e-10, '; erry: ', -2.3699994367021304e-07)\\
('n: ', 23, '; errx: ', -4.4403969390316433e-10, '; erry: ', -2.3699997186987787e-07)\\
('n: ', 24, '; errx: ', -2.3066850651787263e-09, '; erry: ', -2.3699998608073258e-07)\\
('n: ', 25, '; errx: ', -6.0319762518190601e-09, '; erry: ', -2.3699999318615994e-07)\\
('n: ', 26, '; errx: ', -1.3482558181010518e-08, '; erry: ', -2.3699999673887362e-07)\\
('n: ', 27, '; errx: ', -2.8383722705527248e-08, '; erry: ', -2.3699999851523046e-07)\\
('n: ', 28, '; errx: ', -5.8186050200248474e-08, '; erry: ', -2.3699999940340888e-07)\\
('n: ', 29, '; errx: ', -1.1779070430151251e-07, '; erry: ', -2.3699999984749809e-07)\\
('n: ', 30, '; errx: ', -2.3700000006954269e-07, '; erry: ', -2.3700000006954269e-07)\\
  \item 3.B

  Based on the result of question 1, we can get:
\[\begin{array}{l}
{x_{n - 1}} = {}^1{\delta _{n - 1}} \le {2^{k - 1}}\left| {{\delta _0}} \right| + ({2^{k - 1}} - 1) \cdot E\\
{y_{n - 1}} = {}^2{\delta _{n - 1}} \le \frac{1}{{{2^{k - 1}}}}\left| {{{\delta '}_0}} \right| + (2 - \frac{1}{{{2^{k - 2}}}}) \cdot E
\end{array}\]
    In experiment 1, 
  \[{{\delta '}_0} = {}^1{\delta _{n - 1}}\]
  so 
\[\begin{array}{l}
{y_{n - 1}} = {}^2{\delta _{n - 1}} \le \frac{1}{{{2^{k - 1}}}}\left| {{{\delta '}_0}} \right| + (2 - \frac{1}{{{2^{k - 2}}}}) \cdot E\\
 \le \frac{1}{{{2^{k - 1}}}}({2^{k - 1}}\left| {{\delta _0}} \right| + ({2^{k - 1}} - 1) \cdot E) + (2 - \frac{1}{{{2^{k - 2}}}}) \cdot E\\
 = \left| {{\delta _0}} \right| + 3 \cdot (1 - \frac{1}{{{2^{k - 1}}}}) \cdot E
\end{array}\]
  In experiment 2, with same logic we can get \[\left| {{x_{n - 1}}} \right| \le \left| {{\delta _0}} \right| + 3 \cdot ({2^{k - 1}} - 1) \cdot E\]
  
  \end{enumerate}

\end{enumerate}
\end{document}