\documentclass[12pt]{article}
 
\usepackage[margin=1in]{geometry}
\usepackage{amsmath,amsthm,amssymb}
\usepackage{listings}
\usepackage{appendix} 
% * <jclu81@gmail.com> 2018-02-14T20:32:15.777Z:
%
% ^.
\newcommand{\N}{\mathbb{N}}
\newcommand{\R}{\mathbb{R}}
\newcommand{\Z}{\mathbb{Z}}
\newcommand{\Q}{\mathbb{Q}}
 
\newenvironment{theorem}[2][Theorem]{\begin{trivlist}
\item[\hskip \labelsep {\bfseries #1}\hskip \labelsep {\bfseries #2.}]}{\end{trivlist}}
\newenvironment{lemma}[2][Lemma]{\begin{trivlist}
\item[\hskip \labelsep {\bfseries #1}\hskip \labelsep {\bfseries #2.}]}{\end{trivlist}}
\newenvironment{exercise}[2][Exercise]{\begin{trivlist}
\item[\hskip \labelsep {\bfseries #1}\hskip \labelsep {\bfseries #2.}]}{\end{trivlist}}
\newenvironment{problem}[2][Problem]{\begin{trivlist}
\item[\hskip \labelsep {\bfseries #1}\hskip \labelsep {\bfseries #2.}]}{\end{trivlist}}
\newenvironment{question}[2][Question]{\begin{trivlist}
\item[\hskip \labelsep {\bfseries #1}\hskip \labelsep {\bfseries #2.}]}{\end{trivlist}}
\newenvironment{corollary}[2][Corollary]{\begin{trivlist}
\item[\hskip \labelsep {\bfseries #1}\hskip \labelsep {\bfseries #2.}]}{\end{trivlist}}
 
\begin{document}
 
\title{Weekly Homework 5}
\author{Jingcheng Lu 42326170\\ 
cs206p: Scientific Computing}
 
\maketitle
 
\begin{problem}{1}
\text{ }\\
For problem 1-4, the code is displayed in the appendix section.
\end{problem}
\begin{problem}{2}
\text{ }\\
Q: Why don’t we want to try it for exp(x)?\\
A: Cause the image of exp(x) doesn't cross the x-axle, it doesn't have root. There is no x for exp(x) to equal 0.\\
\item Result of Problem 2:\\
For funtion funsin:	 The zero of x occurs at x= 0.000000 	Loop 6 times\\
For funtion funcos:	 The zero of x occurs at x= 1.570796 	Loop 5 times\\
For funtion funtan:	 The zero of x occurs at x= 0.000000 	Loop 6 times\\
For funtion funlog:	 The zero of x occurs at x= 1.000000 	Loop 1 times\\
For funtion funpow2:	 The zero of x occurs at x= 0.000000 	Loop 213 times\\
For funtion funpow3:	 The zero of x occurs at x= 0.000000 	Loop 362 times\\
\end{problem}
\begin{problem}{3}
\text{ }\\
In this problem, I use 3 stop criterion:\\
SC1. number of iterations $>$ MAX-ITERATION-TIMES 
SC2. \[\left | x_{k+1}-x_{k} \right |< \varepsilon \]
SC3. \[ \left | f(x_{k+1})-0 \right |<\varepsilon \]

\item Result of Problem 3:\\
SC1:\\
For Funtion funsin and Stop Criterion 1:	 The zero of x occurs at x= 0.000000 	Loop 100001 times\\
For Funtion funcos and Stop Criterion 1:	 The zero of x occurs at x= 1.570796 	Loop 100001 times\\
For Funtion funtan and Stop Criterion 1:	 The zero of x occurs at x= 0.000000 	Loop 100001 times\\
For Funtion funlog and Stop Criterion 1:	 The zero of x occurs at x= 1.000000 	Loop 100001 times\\
For Funtion funpow2 and Stop Criterion 1:	 The zero of x occurs at x= 0.000000 	Loop 100001 times\\
For Funtion funpow3 and Stop Criterion 1:	 The zero of x occurs at x= 0.000000 	Loop 100001 times\\

SC2:\\
For Funtion funsin and Stop Criterion 2:	 The zero of x occurs at x= 0.000000 	Loop 6 times\\
For Funtion funcos and Stop Criterion 2:	 The zero of x occurs at x= 1.570796 	Loop 5 times\\
For Funtion funtan and Stop Criterion 2:	 The zero of x occurs at x= 0.000000 	Loop 6 times\\
For Funtion funlog and Stop Criterion 2:	 The zero of x occurs at x= 1.000000 	Loop 1 times\\
For Funtion funpow2 and Stop Criterion 2:	 The zero of x occurs at x= 0.000000 	Loop 67 times\\
For Funtion funpow3 and Stop Criterion 2:	 The zero of x occurs at x= 0.000000 	Loop 112 times\\

SC3:\\
For Funtion funsin and Stop Criterion 3:	 The zero of x occurs at x= 0.000000 	Loop 5 times\\
For Funtion funcos and Stop Criterion 3:	 Exceed limit time\\
For Funtion funtan and Stop Criterion 3:	 The zero of x occurs at x= 0.000000 	Loop 5 times\\
For Funtion funlog and Stop Criterion 3:	 The zero of x occurs at x= 1.000000 	Loop 1 times\\
For Funtion funpow2 and Stop Criterion 3:	 The zero of x occurs at x= 0.000000 	Loop 34 times\\
For Funtion funpow3 and Stop Criterion 3:	 The zero of x occurs at x= 0.000000 	Loop 38 times\\

\end{problem}
 
\begin{problem}{4}
\text{ }\\
In this problem, I use the SC2 described above.
\item Result for problem 4:\\
For Funtion funcosp1 and Stop Criterion 2:	 The zero of x occurs at x= 3.141593 	Loop 27 times\\
It takes more times to converge when compared to cos(x). Increase the base make the function's root away from 1, so it's harder to converge.
\end{problem}
 
\begin{problem}{5}
\text{ }\\
In this problem, I use the combination of 3 stop criterion list above.
\item Result for problem 5:\\
For Funtion funpow2:	 The zero of x occurs at x= 0.000000 	Loop 1 times 	 m= 2.000000\\
For Funtion funpow3:	 The zero of x occurs at x= 0.000000 	Loop 1 times 	 m= 3.000000\\
For Funtion funcosp1:	 The zero of x occurs at x= 3.141593 	Loop 5 times 	 m= 2.014195\\
From the result, we can conclude that the number of loop is much smaller then in problem 1 to 4.
\end{problem}

\begin{appendices}  

\section{Code for P1-P2}  
\lstinputlisting[language=Python]{q1-2.py}  
\section{Code for P3-P4}  
\lstinputlisting[language=Python]{q3-4.py}
\section{Code for P5}  
\lstinputlisting[language=Python]{q5.py}
\end{appendices}  
\end{document}
